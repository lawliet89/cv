%% LyX 2.3.2 created this file.  For more info, see http://www.lyx.org/.
%% Do not edit unless you really know what you are doing.
\documentclass[12pt,a4paper,english]{moderncv}
\usepackage[T1]{fontenc}
\usepackage[latin9]{inputenc}
\setcounter{secnumdepth}{0}
\setlength{\parskip}{\medskipamount}
\setlength{\parindent}{0pt}
\usepackage{babel}
\ifx\hypersetup\undefined
  \AtBeginDocument{%
    \hypersetup{unicode=true}
  }
\else
  \hypersetup{unicode=true}
\fi

\makeatletter

%%%%%%%%%%%%%%%%%%%%%%%%%%%%%% LyX specific LaTeX commands.


\moderncvstyle{casual}


\moderncvcolor{red}


\renewcommand{\pdfpagemode}{UseNone}


\name{Yong Wen}{Chua}


\title{}



\email{me@yongwen.xyz}



\social[twitter]{lawlietSG}


\social[linkedin]{yong-wen}


\social[github]{lawliet89}

\special{papersize=\the\paperwidth,\the\paperheight}


%%%%%%%%%%%%%%%%%%%%%%%%%%%%%% User specified LaTeX commands.
%\renewcommand{\familydefault}{\sfdefault}    % to set the default font; use '\sfdefault' for the default sans serif font, '\rmdefault' for the default roman one, or any tex font name
%\nopagenumbers{}                             % uncomment to suppress automatic page numbering for CVs longer than one page

% adjust the page margins
\usepackage[scale=0.75]{geometry}
%\setlength{\hintscolumnwidth}{3cm}           % if you want to change the width of the column with the dates
%\setlength{\makecvtitlenamewidth}{10cm}      % for the 'classic' style, if you want to force the width allocated to your name and avoid line breaks. be careful though, the length is normally calculated to avoid any overlap with your personal info; use this at your own typographical risks...

\makeatother

\begin{document}

\makecvtitle{}{}

\section{Education}

\cventry{2010\textendash 2014}{MEng Electronics and Information Engineering}{Imperial College London}{London}{United Kingdom}{First Class Honours}

\section{Experience}

\cventry{2014\textendash \\
current}{Open Source Contributions}{}{}{}{Please see below.}

\cventry{2019\textendash \\
current}{BasisAI}{Software and DevOps Engineer}{}{Singapore}{%
\begin{minipage}[t]{1\columnwidth}%
\begin{itemize}
\item Design, implement, and manage the infrastructure for our product and
customer using a plethora of open source tools such as Terraform,
Consul, Vault, and Terraform
\item Fully automate the monitoring of the health of the infrastructure
with alerts when issues occur
\end{itemize}
%
\end{minipage}}

\cventry{2015\textendash \\
2019}{Government Technology Agency}{Software Engineer}{}{Singapore}{%
\begin{minipage}[t]{1\columnwidth}%
\begin{itemize}
\item Dockerized applications and set up and Docker based continuous integration
and delivery pipeline using TeamCity for several applications, including
the Business Grant Portal. This allowed easy replication of similar
test and production environment for developers to work with. 
\item Fully automate the deployment of applications using Ansible. 
\item Built an open source JSON Web Token (JWT) based authentication service
for internal micro-services using Rust. In the process, I also wrote
some libraries to support this service including libraries to handle
JWT and Cross-origin resource sharing. 
\item Occasional open source contribution to tools like Docker Compose and
Yarn package manager. 
\item Deployed a Nomad and Consul cluster on AWS in a fully automated manner
using Terraform.
\end{itemize}
%
\end{minipage}}

\cventry{2014\textendash 2015}{LShift. Ltd}{Senior Developer}{London}{United Kingdom}{Worked on several .NET projects and a Java project, including
a MVC web service}

\cventry{2013}{LShift. Ltd}{Summer Internship}{London}{United Kingdom}{Worked on an iPad application and a service for converting Google
Docs to \LaTeX .}

\section{Languages}

\cvitemwithcomment{English}{Native}{}

\cvitemwithcomment{Chinese}{Native}{}

\section{Computer skills}

\cvitem{OS}{Linux, Windows}

\cvitem{Programming}{Rust, Go, C\#, C++, Python, Ruby, JavaScript, Shell}

\cvitem{DevOps tools}{Ansible, Docker, Vagrant, Terraform, Nomad, Consul, Vault}

\cvitem{Methodology}{Agile, Scrum, DSDM, Continuous Integration/Deployment}

\section{Open Source Projects}

\cvitem{}{This is a partial list of open source projects.}

\cvitem{Rust}{\href{https://github.com/lawliet89/biscuit}{JOSE library (https://git.io/v7dIz)}}

\cvitem{Rust}{\href{https://github.com/lawliet89/pr_demon}{Continuous Integration tool for Bitbucket (https://git.io/v7dI2)}}

\cvitem{Rust}{\href{https://github.com/lawliet89/fusionner}{Automated tool for test merging topic branches for Git (https://git.io/v7dIV)}}

\cvitem{Rust}{ \href{https://github.com/lawliet89/rocket_cors}{CORS Library for the Rocket.rs framework (https://git.io/v7dIw)}}

\cvitem{Go}{ \href{https://github.com/terraform-providers/terraform-provider-vault}{Contributor to Terraform Vault Provider (https://git.io/Jeayk)}}

\cvitem{C++}{\href{https://github.com/lawliet89/LLVM-Obfuscator}{LLVM Obfuscator: Implementation of Obfuscation techniques with LLVM compiler (https://git.io/v7dIr)}}

\cvitem{Ruby}{\href{https://github.com/GovTechSG/libcorppass}{Library for Rack applications to intergrate with CorpPass, a SAML based identity provider (https://git.io/v7dIo)}}

\cvitem{Ansible}{ \href{https://github.com/lawliet89/docker-compose-ansible}{Ansible role for deploying Docker applications with Docker Compose (https://git.io/v7dIK)}}
\end{document}
